\documentclass[UTF8, a4paper, 12pt]{ctexart}

%===========================================================
% 宏包导入
%===========================================================
\usepackage{geometry}       % 页面布局
\usepackage{amsmath}        % 数学公式
\usepackage{amsfonts}       % 数学字体
\usepackage{graphicx}       % 图片支持
\usepackage{booktabs}       % 三线表 (用于专业的表格线)
\usepackage{hyperref}       % 超链接
\usepackage{xcolor}         % 颜色支持
\usepackage{listings}       % 代码块支持
\usepackage{float}          % 浮动体控制

%===========================================================
% 格式设置
%===========================================================
% 页面边距设置
\geometry{left=2.54cm,right=2.54cm,top=2.54cm,bottom=2.54cm}

% 超链接颜色设置
\hypersetup{
    colorlinks=true,
    linkcolor=black,
    filecolor=magenta,      
    urlcolor=blue,
    citecolor=black,
    pdftitle={人工智能编程大作业实验报告},
    pdfauthor={董彦嘉}
}

% 代码块样式设置
\definecolor{codegreen}{rgb}{0,0.6,0}
\definecolor{codegray}{rgb}{0.5,0.5,0.5}
\definecolor{codepurple}{rgb}{0.58,0,0.82}
\definecolor{backcolour}{rgb}{0.95,0.95,0.92}

\lstdefinestyle{mystyle}{
    backgroundcolor=\color{backcolour},   
    commentstyle=\color{codegreen},
    keywordstyle=\color{magenta},
    numberstyle=\tiny\color{codegray},
    stringstyle=\color{codepurple},
    basicstyle=\ttfamily\footnotesize, % 使用等宽字体
    breakatwhitespace=false,         
    breaklines=true,                 
    captionpos=b,                    
    keepspaces=true,                 
    numbers=left,                    
    numbersep=5pt,                  
    showspaces=false,                
    showstringspaces=false,
    showtabs=false,                  
    tabsize=4,
    frame=single,                    %哪怕代码很少也显示边框
    rulecolor=\color{black!30}       % 边框颜色
}

\lstset{style=mystyle}

%===========================================================
% 文档头部信息
%===========================================================
\title{\textbf{人工智能编程大作业实验报告:\\Task 3 自定义卷积网络框架与 PyTorch 性能对比}}
\author{学号:2400013169 \quad 姓名:董彦嘉}
\date{2026年1月16日}

%===========================================================
% 正文开始
%===========================================================
\begin{document}

\maketitle

% 摘要
\begin{abstract}
本实验旨在脱离现有的深度学习框架,基于 CUDA、pybind11 和 Python 自主实现一个卷积神经网络框架 \texttt{mytensor}。实验成功实现了卷积层的前向传播与反向传播算子,构建了自动微分机制,并在 CIFAR-10 数据集上完成了图像分类任务。结果显示,自定义框架达到了 54.07\% 的测试准确率。虽然成功验证了底层算法的正确性,但在执行效率(比 PyTorch 慢约 8\%)和收敛稳定性上与工业级框架仍存在差距。本报告详细对比了两者在算子优化、内存管理及数值稳定性方面的差异。
\end{abstract}

% 目录(可选,如不需要可注释掉)
% \tableofcontents
% \newpage

\section{实验背景}
本次实验旨在通过实现 CIFAR-10 数据集的图像分类任务,深入理解深度学习框架的底层原理 [2]。实验分为两个主要部分:
\begin{enumerate}
    \item 使用工业级框架 PyTorch 进行标准实现,作为性能基准(Benchmark)[7]。
    \item 基于 CUDA、pybind11 和 Python 自主实现一个卷积网络框架(\texttt{mytensor}),并完成前向传播、反向传播及自动微分逻辑 [36, 38]。
\end{enumerate}

\section{实验方法与设置}

\subsection{网络架构}
为了公平对比,实验在两个框架中均采用类 LeNet 结构 [37]:
\begin{itemize}
    \item \textbf{卷积层 1}: 3 输入通道, 6 输出通道, $3\times3$ 卷积核, padding=1。
    \item \textbf{池化层 1}: $2\times2$ Max Pooling。
    \item \textbf{卷积层 2}: 6 输入通道, 16 输出通道, $3\times3$ 卷积核, padding=1。
    \item \textbf{池化层 2}: $2\times2$ Max Pooling。
    \item \textbf{全连接层}: 三层 MLP(多层感知机),最终输出 10 类概率。
\end{itemize}

\subsection{训练配置}
\begin{itemize}
    \item \textbf{硬件环境}: 使用支持 CUDA 的 GPU 进行加速训练 [41]。
    \item \textbf{超参数}: Batch Size = 64,Momentum = 0.9。
    \item \textbf{学习率}: 初始学习率 0.01,第 10 轮后衰减至 0.001。
    \item \textbf{数据集}: CIFAR-10 训练集用于参数优化,测试集用于评估准确率。
\end{itemize}

\section{实验结果对比}

根据实际运行输出,汇总性能指标如表 \ref{tab:comparison} 所示:

\begin{table}[htbp]
\centering
\caption{自定义框架 (mytensor) 与 PyTorch 框架性能对比}
\label{tab:comparison}
\begin{tabular}{@{}lccc@{}}
\toprule
\textbf{指标} & \textbf{自定义框架 (mytensor)} & \textbf{PyTorch 框架} & \textbf{差异 (PyTorch 为基准)} \\ \midrule
训练总耗时 (s) & 327.35 & 303.09 & +8.0\% \\
最终测试准确率 & 54.07\% & 66.78\% & -12.71\% \\
收敛平稳性 & 波动较大 & 相对平稳 & - \\ \bottomrule
\end{tabular}
\end{table}

\section{讨论与分析}

\subsection{执行效率分析}
PyTorch 的训练速度比自定义框架快约 24.26 秒(约 8\%)。分析原因如下:
\begin{itemize}
    \item \textbf{算子优化}: PyTorch 底层调用了高度优化的 cuDNN 库,而自定义 CUDA 实现采用了基础的 im2col 算法,在内存访问模式(Memory Access Pattern)和线程块调度上存在提升空间 [43]。
    \item \textbf{内存管理}: PyTorch 拥有成熟的显存池(Caching Allocator)管理机制,减少了频繁申请和释放显存的系统调用开销。
    \item \textbf{数据交互}: 自定义代码中存在 \texttt{inputs.numpy()} 调用,这在每一批次都引入了额外的 CPU-GPU 数据拷贝开销。
\end{itemize}

\subsection{准确率与收敛分析}
PyTorch 最终达到了 66.78\% 的准确率,显著优于自定义框架。
\begin{itemize}
    \item \textbf{权重初始化}: PyTorch 默认使用 Kaiming 或 Xavier 初始化,适配 ReLU 激活函数;而自定义框架使用了简单的随机均匀分布,导致初期训练梯度传递效率较低。
    \item \textbf{数值稳定性}: PyTorch 的 \texttt{CrossEntropyLoss} 在数学上进行了 Log-Sum-Exp 技巧优化,避免了溢出风险;自定义框架的 Loss 值(0.03-0.09)与 PyTorch(0.2-2.3)存在数量级差异,说明 Loss 函数的缩放比例或计算细节不一致。
\end{itemize}

\section{结论}
本实验成功复现了卷积网络在两个不同层级框架上的训练过程 [53]。实验证明,自主实现的 CUDA 算子与自动微分框架能够完成基本的分类任务,但在极端性能优化和数值稳定性方面与成熟框架仍有差距。未来的改进方向应集中在 CUDA kernel 的共享内存(Shared Memory)优化以及更精细的参数初始化策略上。

%===========================================================
% 附录
%===========================================================
\appendix
\section{附录:核心代码实现}

\subsection{CUDA 卷积算子 (im2col 示意)}
\begin{lstlisting}[language=C++]
__global__ void im2col_kernel(const float* data_im, float* data_col, 
                              int channels, int height, int width, 
                              int ksize, int pad, int stride, 
                              int height_col, int width_col) {
    // 计算全局索引
    int index = blockIdx.x * blockDim.x + threadIdx.x;
    const int n = height_col * width_col * channels;
    
    if (index < n) {
        int w_out = index % width_col;
        int h_out = (index / width_col) % height_col;
        int c_in = index / width_col / height_col;
        
        // ... (im2col 坐标映射逻辑) ...
    }
}
\end{lstlisting}

\subsection{Python 自动微分调用}
\begin{lstlisting}[language=Python]
class Conv2D(Module):
    def __init__(self, in_channels, out_channels, ksize):
        # 权重初始化(随机均匀分布)
        self.weight = Tensor.uniform(out_channels, in_channels, ksize, ksize)
    
    def forward(self, x):
        # 调用 C++ 扩展 (mytensor_backend)
        return mytensor_backend.conv2d(x, self.weight)
\end{lstlisting}

\end{document}\documentclass[UTF8, a4paper, 12pt]{ctexart}

%===========================================================
% 宏包导入
%===========================================================
\usepackage{geometry}       % 页面布局
\usepackage{amsmath}        % 数学公式
\usepackage{amsfonts}       % 数学字体
\usepackage{graphicx}       % 图片支持
\usepackage{booktabs}       % 三线表 (用于专业的表格线)
\usepackage{hyperref}       % 超链接
\usepackage{xcolor}         % 颜色支持
\usepackage{listings}       % 代码块支持
\usepackage{float}          % 浮动体控制

%===========================================================
% 格式设置
%===========================================================
% 页面边距设置
\geometry{left=2.54cm,right=2.54cm,top=2.54cm,bottom=2.54cm}

% 超链接颜色设置
\hypersetup{
    colorlinks=true,
    linkcolor=black,
    filecolor=magenta,      
    urlcolor=blue,
    citecolor=black,
    pdftitle={人工智能编程大作业实验报告},
    pdfauthor={董彦嘉}
}

% 代码块样式设置
\definecolor{codegreen}{rgb}{0,0.6,0}
\definecolor{codegray}{rgb}{0.5,0.5,0.5}
\definecolor{codepurple}{rgb}{0.58,0,0.82}
\definecolor{backcolour}{rgb}{0.95,0.95,0.92}

\lstdefinestyle{mystyle}{
    backgroundcolor=\color{backcolour},   
    commentstyle=\color{codegreen},
    keywordstyle=\color{magenta},
    numberstyle=\tiny\color{codegray},
    stringstyle=\color{codepurple},
    basicstyle=\ttfamily\footnotesize, % 使用等宽字体
    breakatwhitespace=false,         
    breaklines=true,                 
    captionpos=b,                    
    keepspaces=true,                 
    numbers=left,                    
    numbersep=5pt,                  
    showspaces=false,                
    showstringspaces=false,
    showtabs=false,                  
    tabsize=4,
    frame=single,                    %哪怕代码很少也显示边框
    rulecolor=\color{black!30}       % 边框颜色
}

\lstset{style=mystyle}

%===========================================================
% 文档头部信息
%===========================================================
\title{\textbf{人工智能编程大作业实验报告:\\Task 3 自定义卷积网络框架与 PyTorch 性能对比}}
\author{学号:2400013169 \quad 姓名:董彦嘉}
\date{2026年1月16日}

%===========================================================
% 正文开始
%===========================================================
\begin{document}

\maketitle

% 摘要
\begin{abstract}
本实验旨在脱离现有的深度学习框架,基于 CUDA、pybind11 和 Python 自主实现一个卷积神经网络框架 \texttt{mytensor}。实验成功实现了卷积层的前向传播与反向传播算子,构建了自动微分机制,并在 CIFAR-10 数据集上完成了图像分类任务。结果显示,自定义框架达到了 54.07\% 的测试准确率。虽然成功验证了底层算法的正确性,但在执行效率(比 PyTorch 慢约 8\%)和收敛稳定性上与工业级框架仍存在差距。本报告详细对比了两者在算子优化、内存管理及数值稳定性方面的差异。
\end{abstract}

% 目录(可选,如不需要可注释掉)
% \tableofcontents
% \newpage

\section{实验背景}
本次实验旨在通过实现 CIFAR-10 数据集的图像分类任务,深入理解深度学习框架的底层原理 [2]。实验分为两个主要部分:
\begin{enumerate}
    \item 使用工业级框架 PyTorch 进行标准实现,作为性能基准(Benchmark)[7]。
    \item 基于 CUDA、pybind11 和 Python 自主实现一个卷积网络框架(\texttt{mytensor}),并完成前向传播、反向传播及自动微分逻辑 [36, 38]。
\end{enumerate}

\section{实验方法与设置}

\subsection{网络架构}
为了公平对比,实验在两个框架中均采用类 LeNet 结构 [37]:
\begin{itemize}
    \item \textbf{卷积层 1}: 3 输入通道, 6 输出通道, $3\times3$ 卷积核, padding=1。
    \item \textbf{池化层 1}: $2\times2$ Max Pooling。
    \item \textbf{卷积层 2}: 6 输入通道, 16 输出通道, $3\times3$ 卷积核, padding=1。
    \item \textbf{池化层 2}: $2\times2$ Max Pooling。
    \item \textbf{全连接层}: 三层 MLP(多层感知机),最终输出 10 类概率。
\end{itemize}

\subsection{训练配置}
\begin{itemize}
    \item \textbf{硬件环境}: 使用支持 CUDA 的 GPU 进行加速训练 [41]。
    \item \textbf{超参数}: Batch Size = 64,Momentum = 0.9。
    \item \textbf{学习率}: 初始学习率 0.01,第 10 轮后衰减至 0.001。
    \item \textbf{数据集}: CIFAR-10 训练集用于参数优化,测试集用于评估准确率。
\end{itemize}

\section{实验结果对比}

根据实际运行输出,汇总性能指标如表 \ref{tab:comparison} 所示:

\begin{table}[htbp]
\centering
\caption{自定义框架 (mytensor) 与 PyTorch 框架性能对比}
\label{tab:comparison}
\begin{tabular}{@{}lccc@{}}
\toprule
\textbf{指标} & \textbf{自定义框架 (mytensor)} & \textbf{PyTorch 框架} & \textbf{差异 (PyTorch 为基准)} \\ \midrule
训练总耗时 (s) & 327.35 & 303.09 & +8.0\% \\
最终测试准确率 & 54.07\% & 66.78\% & -12.71\% \\
收敛平稳性 & 波动较大 & 相对平稳 & - \\ \bottomrule
\end{tabular}
\end{table}

\section{讨论与分析}

\subsection{执行效率分析}
PyTorch 的训练速度比自定义框架快约 24.26 秒(约 8\%)。分析原因如下:
\begin{itemize}
    \item \textbf{算子优化}: PyTorch 底层调用了高度优化的 cuDNN 库,而自定义 CUDA 实现采用了基础的 im2col 算法,在内存访问模式(Memory Access Pattern)和线程块调度上存在提升空间 [43]。
    \item \textbf{内存管理}: PyTorch 拥有成熟的显存池(Caching Allocator)管理机制,减少了频繁申请和释放显存的系统调用开销。
    \item \textbf{数据交互}: 自定义代码中存在 \texttt{inputs.numpy()} 调用,这在每一批次都引入了额外的 CPU-GPU 数据拷贝开销。
\end{itemize}

\subsection{准确率与收敛分析}
PyTorch 最终达到了 66.78\% 的准确率,显著优于自定义框架。
\begin{itemize}
    \item \textbf{权重初始化}: PyTorch 默认使用 Kaiming 或 Xavier 初始化,适配 ReLU 激活函数;而自定义框架使用了简单的随机均匀分布,导致初期训练梯度传递效率较低。
    \item \textbf{数值稳定性}: PyTorch 的 \texttt{CrossEntropyLoss} 在数学上进行了 Log-Sum-Exp 技巧优化,避免了溢出风险;自定义框架的 Loss 值(0.03-0.09)与 PyTorch(0.2-2.3)存在数量级差异,说明 Loss 函数的缩放比例或计算细节不一致。
\end{itemize}

\section{结论}
本实验成功复现了卷积网络在两个不同层级框架上的训练过程 [53]。实验证明,自主实现的 CUDA 算子与自动微分框架能够完成基本的分类任务,但在极端性能优化和数值稳定性方面与成熟框架仍有差距。未来的改进方向应集中在 CUDA kernel 的共享内存(Shared Memory)优化以及更精细的参数初始化策略上。

%===========================================================
% 附录
%===========================================================
\appendix
\section{附录:核心代码实现}

\subsection{CUDA 卷积算子 (im2col 示意)}
\begin{lstlisting}[language=C++]
__global__ void im2col_kernel(const float* data_im, float* data_col, 
                              int channels, int height, int width, 
                              int ksize, int pad, int stride, 
                              int height_col, int width_col) {
    // 计算全局索引
    int index = blockIdx.x * blockDim.x + threadIdx.x;
    const int n = height_col * width_col * channels;
    
    if (index < n) {
        int w_out = index % width_col;
        int h_out = (index / width_col) % height_col;
        int c_in = index / width_col / height_col;
        
        // ... (im2col 坐标映射逻辑) ...
    }
}
\end{lstlisting}

\subsection{Python 自动微分调用}
\begin{lstlisting}[language=Python]
class Conv2D(Module):
    def __init__(self, in_channels, out_channels, ksize):
        # 权重初始化(随机均匀分布)
        self.weight = Tensor.uniform(out_channels, in_channels, ksize, ksize)
    
    def forward(self, x):
        # 调用 C++ 扩展 (mytensor_backend)
        return mytensor_backend.conv2d(x, self.weight)
\end{lstlisting}

\end{document}